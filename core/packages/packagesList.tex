%definition du document
\documentclass[a4paper,oneside,12pt]{article}

%Liste des packages utilises
\usepackage[french]{babel} %prise en charges du francais
\usepackage[utf8]{inputenc} % encodage en entree
\usepackage[T1]{fontenc} % encodage en sortie
\usepackage{graphicx} % inserer des images
\usepackage{wrapfig} % inserer de facon plus libre des images
\usepackage{longtable,geometry} % faire de tableau 
\usepackage{ae,lmodern} % police "moderne" et vectorielle
\usepackage{hyperref} % lien / url
\usepackage{enumitem} % personalisation des listes a puces
\usepackage{fullpage} % eviter la mise en page "livre"
\usepackage{eso-pic} % Uitile pour la mise en page de la page de garde
\usepackage{listings} % mettre en forme du code dans le document
\usepackage{color} % utiliser des couleurs etc...
\usepackage{tikz} % permet de dessiner et faire de l'UML entre autre
\usepackage{core/packages/uml_mod} %personalisation des diagramme UML
\usepackage{csvsimple} % tableau depuis csv
\usepackage{booktabs} % stylisation de tableau
\usepackage{caption} % description élément
\usepackage[most]{tcolorbox} % faire des box

% utile la stylisation du header et du footer
\usepackage{fancyhdr}
\usepackage{titlesec}
\usepackage{etoolbox}

