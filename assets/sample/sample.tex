\input{inc/packages.tex}
\definecolor{codegreen}{rgb}{0,0.6,0}
\definecolor{codegray}{rgb}{0.5,0.5,0.5}
\definecolor{backcolour}{rgb}{0.95,0.95,0.92}

\definecolor{codeorange}{RGB}{245,134,34}
\definecolor{codeblue}{RGB}{10,125,224}
 
\lstdefinestyle{mystyle}{
    backgroundcolor=\color{backcolour},   
    commentstyle=\color{codegreen},
    keywordstyle=\color{codeblue},
    numberstyle=\tiny\color{codegray},
    stringstyle=\color{codeorange},
    basicstyle=\footnotesize,
    breakatwhitespace=false,         
    breaklines=true,                 
    captionpos=b,                    
    keepspaces=true,                 
    numbers=left,                    
    numbersep=5pt,                  
    showspaces=false,                
    showstringspaces=false,
    showtabs=false,                  
    tabsize=2
}

\lstdefinestyle{codeBloc}{
    commentstyle=\color{codegreen},
    keywordstyle=\color{codeblue},
    numberstyle=\tiny\color{codegray},
    stringstyle=\color{codeorange},
    breaklines=true,
}

%\lstset{style=mystyle} %optionnel
%%%%%%%%%% GENERAL %%%%%%%%%%
%%%%%%%%%%%%%%%%%%%%%%%%%%%%%%
\def\date{\today}
\RequirePackage[english]{babel}% define language of document (english by default) — babel is the name of the package
\def\coversheet{true}% false if you don't want a coversheet

%%%%%%%%%%% COLOR %%%%%%%%%%%
%%%%%%%%%%%%%%%%%%%%%%%%%%%%%
\definecolor{mainColor}{HTML}{315399} % document main color (blue by default)
\definecolor{textColorHeader}{HTML}{C1C9D1} % color text header (light grey by default)

%%%%%%%%%%% IMAGE %%%%%%%%%%%%
%%%%%%%%%%%%%%%%%%%%%%%%%%%%%%
\def\headerImage{assets/img/paper-clip.png} % header image of document
\def\headerImageScale{.2} % sclale of header image (0.2 by default)
\def\coverSheetImage{assets/img/background.png} % image coverpage
\def\logoImage{assets/img/paper-clip.png} % logo on cover page 

%%%%%%%%%%% INFOS %%%%%%%%%%%%%
%%%%%%%%%%%%%%%%%%%%%%%%%%%%%%%
\def\title{Document Title} % the title of the document
\def\author{Document Author} % the author of the document
\def\compagny{Compagny Name}
\def\copyright{%  the copyright of the document, by default CC-BY-CC — leave it empty if don't need it
    {\tiny %
    \title,~\the\year{},~\compagny,~by \author~is under licence :~%
    \href{https://creativecommons.org/licenses/by-nc/4.0/?ref=chooser-v1}{\faCreativeCommons~\faCreativeCommonsBy~\faCreativeCommonsNc}
    }%
}

\enlargethispage{2cm} %reduire marge de bas de page

\newcommand\BackgroundPic{%
\put(0,0){%
\parbox[b][\paperheight]{\paperwidth}{%
\vfill
\centering
\includegraphics[width=\paperwidth,height=\paperheight,%
keepaspectratio]{\BackgroundTitlePage}%
\vfill
}}}

\patchcmd{\chapter}{\thispagestyle{plain}}{\thispagestyle{fancy}}{}{}

\definecolor{blueCustom}{RGB}{41,152,205}
\definecolor{redCustom}{RGB}{205,41,41}
\definecolor{greenCustom}{RGB}{119,205,41}
\definecolor{orangeCustom}{RGB}{230,129,62}
\definecolor{textcolor}{RGB}{240,240,240}

\newtcolorbox{info}{
    enhanced,sharp corners,
    width={16cm},
    enlarge top by=0.3cm,
    colframe=blueCustom,
    colback=white,
    overlay={\node at (frame.north west) {\includegraphics[scale=0.1]{img/info.png}};}
}

\newtcolorbox{warning}{
    enhanced,sharp corners,
    width={16cm},
    enlarge top by=0.3cm,
    colframe=redCustom,
    colback=white,
    overlay={\node at (frame.north west) {\includegraphics[scale=0.1]{img/warning.png}};}
}

\newtcolorbox{advice}{
    enhanced,sharp corners,
    width={16cm},
    enlarge top by=0.3cm,
    colframe=greenCustom,
    colback=white,
    overlay={\node at (frame.north west) {\includegraphics[scale=0.1]{img/advice.png}};}
}

\newtcolorbox{definition}{
    enhanced,sharp corners,
    width={16cm},
    enlarge top by=0.3cm,
    colframe=orangeCustom,
    colback=white,
    overlay={\node at (frame.north west) {\includegraphics[scale=0.1]{img/def.png}};}
}


%\lstset{style=codeBloc} % for command shell display only
\lstset{style=mystyle} % normal code display
\newtcblisting{commandshell}{colback=black,colupper=white,colframe=black!75!black,
listing only,listing options={style=codeBloc},
every listing line={\textcolor{greenCustom}{\small\ttfamily\bfseries user@desktop > }}}

\usetikzlibrary{calc}
\renewcommand{\headrulewidth}{0pt}

\pagestyle{fancy}
\fancyhf{}
\fancyhead[C]{%
\begin{tikzpicture}[overlay, remember picture]%
    \fill[\Color] (current page.north west) rectangle ($(current page.north east)+(0,-1in)$);
    \node[anchor=north west, text=textcolor, font=\scshape, minimum size=1in, inner xsep=5mm] at (current page.north west) {\titre};
    \node[anchor=north east, minimum size=1in, inner xsep=5mm] at (current page.north east) {\includegraphics[scale=.2]{\ImageRightPage}};
      %node[minimum width=\x2-\x1, minimum height=2cm, draw, rectangle, fill=blueCustom!20, anchor=north west, align=left, text width=\x2-\x1] at ($(current page.north west)$) {\Large\bfseries \quad #1};
\end{tikzpicture}
}

\fancyfoot[CO]{
    \begin{tikzpicture}[overlay, remember picture]%
    %\node[anchor=south west, opacity=0.5] at (current page.south west) {\includegraphics[scale=.3]{img/img_bottom.png}};
    % \node[anchor=south east, text=black, font=\large, minimum size=.10in] at (current page.south east) {\thepage};  % left
    \node[midway, text=black, font=\large] at (current page.south) {\thepage}; %center 
\end{tikzpicture}
}

\setlength{\headheight}{12pt}

\title{\Huge{\nom}}
\author{\auteur}

\makeatletter 

\begin{document}
    \AddToShipoutPicture*{\BackgroundPic}% mettre l'image de fond
    \begin{titlepage}

    \begin{multicols}{2}{
        
        \begin{flushleft}
            \includegraphics[scale=0.3]{\logoEntreprise}
        \end{flushleft}
        
        \begin{flushright}
            \huge{\textbf{\annee}}
        \end{flushright}
    }
    \end{multicols}

    \vspace*{3cm}

    \begin{flushright}
        \huge{\titre}\\

        \normalsize{\textsc{\nomEntreprise~-~\lieuEntreprise}}
    \end{flushright}

    \vspace*{10cm}

    \begin{multicols}{2}{
      
    \begin{flushleft}
        \vspace*{4cm}
        \includegraphics[width = 4cm, height= 2cm]{\logoIUT}
    \end{flushleft}

    \begin{flushright}
        \auteur  \\\infoEtudiant

        Tuteur de stage : \tuteurStage  \\ Maître de stage : \maitreDeStage
    \end{flushright}

    }
    \end{multicols}

    

\end{titlepage}

 % inclure la page de garde
    \ClearShipoutPicture
    

	\hypersetup{pdfborder=0 0 0} % enlever encadrement des liens internes
	

    \newpage
    \thispagestyle{empty}
    \mbox{}
    \newpage
	
	
    \renewcommand{\thepage}{page \arabic{page}}
    \renewcommand{\thesubsubsection}{\alph{subsubsection})} 
    %enleve la numerotation sur la premiere page (page de garde)	
	\setcounter{page}{1}
	

    \tableofcontents % Fait la table des matiere
    \newpage % change de page
	
%%  START %%

	\section{UML example}

 	\begin{figure}[h]

	\begin{tikzpicture}

	  \begin{class}[text width=8cm]{ClassName}{0,0}
	    \attribute{name : attribute type}
	    \attribute{name : attribute type = default value}
	
	    \operation{name(parameter list) : type of value returned}
	    % virtual operation
	    \operation[0]{name(parameters list) : type of value returned}
	  \end{class}

	\end{tikzpicture}

	  \caption{une Classe UML classsique.}
	  \centering
	\end{figure}


	\begin{figure}[h]

	\begin{tikzpicture}

	  \begin{abstractclass}[text width=5cm]{BankAccount}{0,0}
	    \attribute{owner : String}
	    \attribute{balance : Dollars = 0}
	    \operation{deposit(amount : Dollars)}
	    \operation[0]{withdrawl(amount : Dollars)}
	  \end{abstractclass}

	\end{tikzpicture}

	  \caption{une Classe abstraite.}
	  \centering
	\end{figure}

	\begin{figure}[h]
	\begin{tikzpicture}
	  \begin{class}{Car}{0,0}
	  \end{class}
	  \begin{class}{Wheel}{7.5,0}
	  \end{class}
	  \aggregation{Car}{ }{4}{Wheel}
	\end{tikzpicture}
	  \caption{une Agrégation.}
	  \centering
	\end{figure}
	

	\begin{figure}[h]
	\begin{tikzpicture}

	  \begin{class}{Company}{0,0}
	  \end{class}

	  \begin{class}{Department}{10,0}
	  \end{class}

	  \composition{Company}{ }{1..*}{Department}
	\end{tikzpicture}

	  \caption{une composiion.}
	  \centering
	\end{figure}

	\begin{figure}[h]
	\begin{tikzpicture}

	  \begin{class}[text width=7cm]{Flight}{0,0}
	    \attribute{flightNumber : Integer}
	    \attribute{departureTime : Date}
	    \attribute{flightDuration : Minutes}
	    \attribute{departingAirport : String}
	    \attribute{arrivingAirport : String}
	
	    \operation{delayFlight ( numberOfMinutes : Minutes )}
	    \operation{getArrivalTime ( ) : Date}
	  \end{class}
	
	  \begin{class}{Plane}{11,0}
	    \attribute{airPlaneType : String}
	    \attribute{maximumSpeed : MPH}
	    \attribute{maximumDistance : Miles}
	    \attribute{tailID : String}
	  \end{class}
	
	  \association{Plane}{assignedPlane}{0..1}{Flight}{0..*}{assignedFlights}
	
	\end{tikzpicture}
	  \caption{une Association.}
	  \centering
	\end{figure}


	\begin{figure}[h]
	\renewcommand{\umltextcolor}{red}
	\renewcommand{\umlfillcolor}{green}
	\renewcommand{\umldrawcolor}{blue}
	
	\begin{tikzpicture}
	  \begin{class}[text width=8cm]{ClassName}{0,0}
	    \attribute{name : attribute type}
	    \attribute{name : attribute type = default value}
	
	    \operation{name(parameter list) : type of value returned}
	    % virtual operation
	    \operation[0]{name(parameters list) : type of value returned}
	  \end{class}
	\end{tikzpicture}

	  \caption{Changer les couleurs.}
	  \centering
	\end{figure}

	\begin{figure}[h]
	\begin{tikzpicture}[show background grid]
	  \begin{interface}{Person}{0,0}
	    \attribute{firstName : String}
	    \attribute{lastName : String}
	  \end{interface}
	
	  \begin{class}{Professor}{-5,-5}
	    \implement{Person}
	    \attribute{salary : Dollars}
	  \end{class}
	
	  \begin{class}{Student}{5,-5}
	    \implement{Person}
	    \attribute{major : String}
	  \end{class}
	\end{tikzpicture}

	  \caption{Implémentation - interface.}
	  \centering
	\end{figure}

	\begin{figure}[h]
	\begin{tikzpicture}[show background grid]
	  \begin{class}[text width=5cm]{BankAccount}{0,0}
	    \attribute{owner : String}
	    \attribute{balance : Dollars = 0}
	
	    \operation{deposit(amount : Dollars)}
	    \operation[0]{withdrawl(amount : Dollars)}
	  \end{class}
	
	  \begin{class}[text width=7cm]{CheckingAccount}{-5,-5}
	    \inherit{BankAccount}
	    \attribute{insufficientFundsFee : Dollars}
	
	    \operation{processCheck ( checkToProcess : Check )}
	    \operation{withdrawal ( amount : Dollars )}
	  \end{class}
	
	  \begin{class}[text width=7cm]{SavingsAccount}{5,-5}
	    \inherit{BankAccount}
	    \attribute{annualInteresRate : Percentage}
	
	    \operation{depositMonthlyInterest ( )}
	    \operation{withdrawal ( amount : Dollars )}
	  \end{class}
	
	\end{tikzpicture}

	  \caption{Héritage.}
	  \centering
	\end{figure}


	\begin{figure}[h]
	\begin{tikzpicture}[show background grid]
	    \begin{class}[text width = 2cm]{TArg}{0, 0}
	    \end{class}
	
	    \begin{class}[text width = 2cm]{TGroup}{5, 0}
	    \end{class}
	
	    \begin{class}[text width = 2cm]{TProgInit}{10, 0}
	    \end{class}
	
	    \begin{class}[text width = 2cm]{TProgram}{5, -2}
	        \inherit{TProgInit}
	        \inherit{TGroup}
	        \inherit{TArg}
	    \end{class}
	\end{tikzpicture}

	  \caption{Multi-Héritage.}
	  \centering
	\end{figure}

	
	\begin{figure}[h]
	\begin{tikzpicture}
	  \umlnote (note) {This is a note.};
	\end{tikzpicture}

	  \caption{Note.}
	  \centering
	\end{figure}

	\begin{figure}[h]
	\begin{tikzpicture}
	  \begin{object}[text width=6cm]{Instance Name}{0,0}
	    \instanceOf{Class Name}
	    \attribute{attribute name = value}
	  \end{object}
	\end{tikzpicture}

	  \caption{Objet.}
	  \centering
	\end{figure}

	\begin{figure}[h]
	\begin{tikzpicture}
	  \begin{object}[text width=7cm]{Thomas' account}{0,0}
	    \instanceOf{BankAccount}
	    \attribute{owner = Thomas}
	    \attribute{balance = 100}
	
	    \operation{deposit(amount : Dollars)}
	    \operation[0]{withdrawl(amount : Dollars)}
	  \end{object}
	\end{tikzpicture}

	  \caption{Objet avec méthodes .}
	  \centering
	\end{figure}

	\begin{figure}[h]
	\begin{tikzpicture}
	  \begin{package}{Accounts}
	    \begin{class}[text width=5cm]{BankAccount}{0,0}
	      \attribute{owner : String}
	      \attribute{balance : Dollars = 0}
	
	      \operation{deposit(amount : Dollars)}
	      \operation[0]{withdrawl(amount : Dollars)}
	    \end{class}
	
	    \begin{class}[text width=7cm]{CheckingAccount}{-5,-5}
	      \inherit{BankAccount}
	      \attribute{insufficientFundsFee : Dollars}
	
	      \operation{processCheck ( checkToProcess : Check )}
	      \operation{withdrawal ( amount : Dollars )}
	    \end{class}
	
	    \begin{class}[text width=7cm]{SavingsAccount}{5,-5}
	      \inherit{BankAccount}
	      \attribute{annualInteresRate : Percentage}
	
	      \operation{depositMonthlyInterest ( )}
	      \operation{withdrawal ( amount : Dollars )}
	    \end{class}
	  \end{package}

	\end{tikzpicture}

	  \caption{Package.}
	  \centering
	\end{figure}

%% END %%
\end{document}
